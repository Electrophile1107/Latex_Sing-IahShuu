\setcounter{page}{1}

\sChapter{}{}{}{mô-t'a}{djün foh-ing shü}{}

\section{Mt.}%1

\begin{sAbstract}
	\item[1] Kyi-toh-keh kô-pu jong Ô-pah-læh-hön tao Iah-seh.
	\item[8] Mô-li-ô bi Sing-Ling kön-dong wa-ying.
	\item[19] T'in-s ao Iah-seh feh yüong p'ô; ka-shih Kyi-toh-keh 'æn-deo.
\end{sAbstract}

\header
\lettrine{Ô}{-pah-læh-hön} 'eo-de, Da-bih-keh 'eo-de, Yia-su Kyi-toh-keh kô-pu.
\par
\hspace{1ex}
+	Ô-pah-læh-hön sang Yi-sæh; Yi-sæh sang Ngô-kôh; Ngô-kôh sang Yiu-da teh ge hyüong-di;
+	Yiu-da jong Da-mô sang Fæh-læh-z teh Sæh-læh; Fæh-læh-z sang Yi-z-leng; Yi-z-leng sang O-læn;
+	O-læn sang Ô-mi-nô-dæh; Ô-mi-nô-dæh sang Nô-jing; Nô-jing sang Sæh-meng;
+	Sæh-meng jong Læh-'eh sang Po-z; Po-z jong Lu-teh sang Ô-pah; Ô-pah sang Yia-si;
+	Yia-si sang Da-bih wông. \par Da-bih wông jong U-li-ô-keh ts'i-ts sang Su-lo-meng;
+	Su-lo-meng sang Lo-po-ön; Lo-po-ön sang Ô-pi-ô; Ô-pi-ô sang Ô-sæh;
+	Ô-sæh sang Iah-sô-fæh; Iah-sô-fæh sang Iah-læn; Iah-læn sang U-si-ô;
+	U-si-ô sang Iah-tæn; Iah-tæn sang Ô-hah-z; Ô-hah-z sang Hyi-si-kô;
+	Hyi-si-kô sang Mô-nô-si; Mô-nô-si sang O-meng; O-meng sang Iah-si-ô;
+	Pah-sing lo-liah tao Pô-pi-leng z-'eo, Iah-si-ô sang Yia-ko-nyi-ô teh ge hyüong-di.
\par
+	Pah-sing lo-liah tao Pô-pi-leng ts-'eo, Yia-ko-nyi-ô sang Sæh-læh-t'ih; Sæh-læh-t'ih sang Su-lo-pô-pah;
+	Su-lo-pô-pah sang Ô-pi-üoh; Ô-pi-üoh sang Yi-li-ô-kying; Yi-li-ô-kying sang Ô-su;
+	Ô-su sang Sæh-dôh; Sæh-dôh sang Ô-kying; Ô-kying sang Yi-leh;
+	Yi-leh sang Yi-li-ô-sæh; Yi-li-ô-sæh sang Mô-dæn; Mô-dæn sang Ngô-kôh;
+	Ngô-kôh sang Iah-seh, ziu-z Mô-li-ô-keh dziang-fu; Mô-li-ô sang Yia-su ts'ing-hwu Kyi-toh.
\par
+	Jong Ô-pah-læh-hön tao Da-bih ih-gyüong yiu zih-s de; jong Da-bih tao pah-sing lo-liah tao Pô-pi-leng z-'eo, yi yiu zih-s de; jong pah-sing lo-liah tao Pô-pi-leng ts-'eo tao Kyi-toh yi yiu zih-s de.
\par
+	\textsc{Yia-su} Kyi-toh sang-keh z-kön z tsiao 'ô-veng: Ge nyiang Mô-li-ô yi-kying hyü peh Iah-seh, wæ-vong tso ts'ing, bi Sing-Ling kön-dong wa-ying.
+	Ge dziang-fu Iah-seh z tsing-dzih nying, feh k'eng ming-ming peh ge tao me, siang s-'ô t'e-hweng.
+	Ge k'eo-k'eo s-ts'eng keh z-kön, Cü-keh t'in-s mong-cong yin-c'ih-le, teh ge kông, ``Da-bih-keh 'eo-de Iah-seh, ng k'e c'ü ng-keh ts'i-ts Mô-li-ô ku-le, feh yüong p'ô: ing-yü ge wa-ying z bi Sing-Ling kön-dong.
+	Ge we sang ih-ke N-ts; ng iao c'ü Ge ming-z \textsc{Yia-su}; ing-yü z Ge iao kyiu Ge pah-sing t'eh-c'ih ge-keh ze-ôh.''
\par
+	Keh z-kön tu tso-zing, hao peh Cü t'ôh sin-ts-nying su kông shih-wa yiu ing-nyin, ziu-z,
+	``Yiu ih-ke do-nyiang we wa-ying sang N-ts, nying iao ts'ing-hwu Ge ming-z \textsc{Yi-mô-ne-li};'' [Y. 7. 14.] fæn-c'ih-le ziu-z, \textsc{Zông-ti teh ngô-he jü-de}.
\par
+	Iah-seh kw'eng-sing nang-k'yi, ziu i Cü t'in-s su feng-fu, c'ü ge ts'i-ts ku-le;
+	dæn-z vong teh ge dong-vông teng-tao ge sang deo ih-ke N-ts: c'ü Ge ming-z \textsc{Yia-su}.



\section{Mt.}%2

\begin{sAbstract}
	\item[1] Pôh-z bi ih-leh sing ying-dao tao Kyi-toh-keh di-fông.
	\item[11] Pa Ge, hyin su ta-le-keh li-veh peh Ge.
	\item[14] Iah-seh ta Yia-su teh Ge nyiang dao tao Yi-gyih.
	\item[16] Hyi-leh wông sæh si-lao.
	\item[20] Hyi-leh s-gao.
	\item[22] Iah-seh ta Yia-su teh Ge-nyiang cün-le, djü ze Kô-li-li Nô-sæh-leh.
\end{sAbstract}

\header
\lettrine{H}{yi-leh} wông z-'eo, Yia-su sang ze Yiu-t'a-keh Pah-li-'eng, yiu kyi-ke pôh-z jong tong-pin tseo tao Yia-lu-sæh-lang, kông,
+	``Sang-c'ih tso Yiu-t'a nying-keh Wông-ti ze 'ah-yi? ing-yü ngô-he ze tong-pin mông-djôh Ge-keh sing-siu, ziu le pa Ge.''
\par
+	Hyi-leh wông t'ing-djôh ziu p'ô-gyü, Yia-lu-sæh-lang-keh cong-nying yi pô-gyü.
+	Ge ziu dziao-le cong tsi-s-tsiang teh pah-sing cong-yiang-keh doh-shü-nying, bön-meng ge Kyi-toh ing-ke sang ze 'ah-yi.
+	Ge-he we-teh kông, ``Ze Yiu-t'a-keh Pah-li-'eng: ing-yü sin-ts-nying yiu sia-lôh, kông,
+	`Yiu-da di-fông-keh Pah-li-'eng, ng ze Yiu-da fu-yün ts-cong bing-fi ting siao: ing-yü jong ng iao c'ih ih-ke kyüing-wông, moh-yiang Ngô Yi-seh-lih pah-sing.' '' [Mi. 5. 2.]
\par
+	Hyi-leh ziu s-'ô ao pôh-z le, ts-si bön-meng ge keh sing-siu c'ih-yin z-'eo.
+	Ziu ts'a ge tao Pah-li-'eng, kông, ``Ng-he k'e ts-si tang-t'ing keh Si-lao z-kön; zing-djôh-ku, ziu le t'ong-pao ngô, peh ngô ah hao k'e pa Ge.''
+	Ge-he t'ing koh-wông shih-wa ziu k'e; ze tong-pin mông-djôh-keh sing-siu ziu ying-dao ge-he, ih-dzih tao Si-lao su-ze-keh zông-deo, ziu ding-lao.
+	Pôh-z mông-djôh keh sing-siu, ting kao-hying.
+	Tseo-tsing oh-li, mông-djôh Si-lao teh Ge nyiang Mô-li-ô; ziu p'oh-lôh pa keh Si-lao; ky'iao-k'e pao-'æh do kying-ts, jü-hyiang, meh-yiah, hyin peh Ge tso li-veh.
+	Pôh-z mong-cong teh-djôh ts-ying, ao ge feh-k'o cün-k'e kyin Hyi-leh, ziu jong bih-diao lu cün tao z-kyi di-fông.
\par
+	Cün-k'e ts-'eo, Cü-keh t'in-s mong-cong yin-c'ih teh Iah-seh kông, ``Nang-ky'i, ta keh Si-lao teh Ge nyiang dao tao Yi-gyih k'e, djü ze-kæn teng-tao ngô t'ong-ts ng: ing-yü Hyi-leh iao zing keh Si-lao djü-mih Ge.''
+	Iah-seh ziu nang-ky'i, yia-li ta keh Si-lao teh Ge nyiang tao Yi-gyih k'e;
+	djü ze-kæn ih-dzih tao Hyi-leh ling-cong: hao peh Cü t'ôh sin-ts-nying su kông shih-wa yiu ing-nyin, ziu-z, ``Ngô jong Yi-gyih ao Ngô N-ts c'ih-le.'' ['O. 11. 1.]
\par
+	Hyi-leh ih hyiao-teh bi pôh-z hong-p'in, ky'i-teh-mang, ziu tsiao ge bön-meng pôh-z z-'eo ts'a nying tao Pah-li-'eng, teh s-hyiang, liang-shü yi-'ô si-lao tu sæh wön.
+	Keh-ts'iah sin-ts Yia-li-mi su kông shih-wa yiu ing-nyin, ziu-z,
+	``Ze Leh-mô t'ing-djôh k'oh-sing, ting sing-t'ong-keh sing-hyiang, Leh-kyih k'oh ge n-ts; feh k'eng ziu ön-yü, ing-yü ge n-ts m-gao.'' [Yl. 31. 15.]
\par
+	Hyi-leh s ts-'eo, yiu Cü-keh t'in-s ze Yi-gyih mong-cong yin-c'ih teh Iah-seh kông,
+	``Nang-ky'i ta keh Si-lao teh Ge nyiang, tao Yi-seh-lih di-fông k'e: ing-yü iao meo-'e Si-lao sing-ming cü-ts yi-kying s-gao.''
+	Iah-seh ziu nang-ky'i, ta keh Si-lao teh Ge nyiang tao Yi-seh-lih di-fông.
+	T'ing-djôh Ô-kyi-lao tsih ge pang Hyi-leh-keh yü tso Yiu-t'a koh-wông, feh kön tao keh di-fông k'e; mong-cong yi teh-djôh ts-ying, ziu cün-wæn tao Kô-li-li di-ka k'e,
+	djü ze Nô-sæh-leh zing-li: hao peh sin-ts-nying su kông shih-wa yiu ing-nyin, ziu-z, ``Nying we ts'ing-hwu Ge Nô-sæh-leh nying.''



\section{Mt.}%3

\begin{sAbstract}
	\item[1] Iah-'ön djün-dao; 'ang si-li.
	\item[7] P'i-bing Fæh-li-se nying,
	\item[13] ze Iah-dæn 'o 'ang si-li peh Kyi-toh.
\end{sAbstract}

\header
\lettrine{K}{eh} z-'eo 'ang si-li Iah-'ön tao Yiu-t'a hông-yia djün-dao, kông,
+	``Ng-he yüong hwe-ke; ing-yü t'in-koh z gying-gao.''
+	Keh ziu-z sin-ts Yi-se-ô su kông, ``Ze hông-yia yiu nying-keh sing-ing ao-hyiang, kông, `Bi-bæn Cü-keh lu, p'u-dzih Ge-keh ka-dao.' '' [Y. 40. 3.]
+	Iah-'ön tsiah-keh i-zông z lôh-do mao, kyi-iao-ta z bi; ky'üoh-zih z wông-djong teh yia-mih.
\par
+	Keh z-'eo Yia-lu-sæh-lang, t'ong Yiu-t'a, teh Iah-dæn 'o-pin-keh nying tu tseo-c'ih tao ge su-ze;
+	tsiao-nying z-keh ze, ze Iah-dæn 'o ziu ge-keh si-li.
\par
+	Iah-'ön mông-djôh hyü-to Fæh-li-se nying teh Sæh-t'u-ke nying le ziu ge-keh si-li, teh ge-he kông, ``Doh-zô-keh cong-tông. Kæh-nying ts-ying ng to-bi tsiang-le-keh da-nu?
+	Su-i iao kyih hwe-ke-keh ko-ts;
+	feh iao sing-li kông, `Ô-pah-læh-hön z ngô-he-keh tsu-tsong:' ngô teh ng kông, Ziu-z keh-sih zih-deo Zông-ti ah neng-keo peh ge tso Ô-pah-læh-hön-keh ts-seng.
+	Yin-ze fu-deo yi-kying fông ze jü keng: væn-pah feh kyih hao ko-ts-keh jü, ziu iao tsôh-lôh tön ze ho-li.
+	Ngô yüong shü 'ang hwe-ke-keh si-li peh ng: dæn-z 'eo-jü ngô le-keh neng-kön tse do-jü ngô, ziu-z do Ge 'a ngô feh kön-tông; Ge we yüong Sing-Ling teh ho, 'ang si-li peh ng:
+	Ge siu k'ô dön-kyi, we pe kön-zing sô-dziang-keh koh: Ge-keh koh we siu-tsing koh-ts'ông, dæn-z koh-hæn yüong fe u-keh ho siao-gao.''
\par
+	\textsc{Keh} z-'eo Yia-su jong Kô-li-li tseo tao Iah-dæn 'o tao Iah-'ön su-ze, iao ziu ge si-li.
+	Iah-'ön t'e-lao, kông, ``Ngô ing-ke ziu Ng-keh si-li, Ng fæn-cün tao ngô su-ze le?''
+	Yia-su we-teh ge kông, ``Yin-ze ts'ia hyü Ngô: ing-yü ngô-he ing-ke z-t'ih tso-zing ih-ts'ih-keh kong-nyi.'' Ge ziu ing-hyü.
+	Yia-su ziu-ku si-li, ziu jong shü-li tseo-zông: t'in k'e peh Ge, ziu mông-djôh Zông-ti-keh Ling ziang bôh-kön kông-ling, ze Ge deo-zông;
+	yi jong t'in yiu sing-ing, kông, ``Keh-z Ngô ts'ing-e-keh N-ts, Ngô ting cong-i-keh.''



\section{Mt.}%4

\begin{sAbstract}
	\item[1] Kyi-toh kying-zih, bi mo-kyü s-lin.
	\item[11] T'in-s le voh-z Ge,
	\item[13] djü ze Kô-pah-nong,
	\item[17] djün dao-li ky'i,
	\item[18] dziao-le Pi-teh teh Ön-teh-lih,
	\item[21] Ngô-kôh teh Iah-'ön,
	\item[23] i yiang-yiang bing-t'ong.
\end{sAbstract}

\header
\lettrine{K}{eh-ts'iah} Yia-su bi Sing-Ling ta tao hông-yia peh mo-kyü s-s Ge.
+	Ge kying-zih s-zih nyih-yia, 'eo-deo z du-hæh.
+	Keh s-s cü-ts tseo-le teh Ge kông, ``Ng ziah-z Zông-ti-keh N-ts, hao feng-fu keh-sih zih-deo pin tso ping.''
+	Yia-su we-teh kông, ``Yiu sia-lôh, `Nying feh tæn-tsih k'ao-djôh ping we weh, z k'ao-djôh Zông-ti k'eo-li ih-ts'ih su kông shih-wa.' '' [Sm. 8. 3.]
\par
+	Mo-kyü ziu ta Ge tao sing-zing-li; peh Ge gyi ze sing-din-keh oh-ting,
+	teh Ge kông, ``Ng ziah-z Zông-ti-keh N-ts, hao t'iao-lôh, ing-yü yiu sia-lôh, `Ge we feng-fu Ge-keh t'in-s tsiao ku Ng: ge-he siu we vu-zu Ng, sang-teh Ng-keh kyiah bang-djôh zih-deo.' '' [S. 91. 11, 12.]
+	Yia-su teh ge kông, ``Yi yiu sia-lôh, `Ng feh-k'o s-s Cü ng-keh Zông-ti.' '' [Sm. 6. 16.]
\par
+	Mo-kyü tse ta Ge tao ting kao sæn-deo, ts-tin t'in-'ô væn-koh teh ge yüong-wa peh Ge mông;
+	teh Ge kông, ``Ng ziah-z p'oh-lôh pa ngô, ngô tu we peh Ng.''
+	Yia-su teh ge kông, ``Sæh-dæn, tseo-k'e: ing-yü yiu sia-lôh, `Ing-ke pa Cü ng-keh Zông-ti, tæn-tsih voh-z Ge.' '' [Sm. 6. 13.]
+	Mo-kyü li-k'e Yia-su: ziu yiu t'in-s le voh-z Ge.
\par
+	\textsc{Yia-su} t'ing-djôh Iah-'ön lôh kæn-lao, ziu tao Kô-li-li k'e;
+	li-k'e Nô-sæh-leh, k'e djü ze he-pin-keh Kô-pah-nong, ziu-z ze Si-pu-leng teh Næh-da-li-keh di-ka:
+	hao peh sin-ts Yi-se-ô su kông shih-wa yiu ing-nyin, ziu-z,
+	``Si-pu-leng teh Næh-da-li di-fông, ze Iah-dæn 'o-nga he-pin, bih-koh nying-keh Kô-li-li,
+	zo ze moh-heh-keh pah-sing mông-djôh do liang-kwông, zo ze heh-ön di-fông ziang ing-s cü-ts, yiu liang-kwông tsiao-djôh ge.'' [Y. 9. 1, 2.]
+	Jong keh z-'eo Yia-su djün dao kông, ``Ng-he yüong hwe-ke; ing-yü t'in-koh z gying gao.''
\par
+	Yia-su ze Kô-li-li he-pin tseo, mông-djôh liang hyüong-di, ziu-z Si-meng yi ts'ing-hwu Pi-teh, teh ge hyüong-di Ön-teh-lih, ze he-li tang mông; ing-yü z k'ô ng-keh.
+	Ziu teh ge-he kông, ``Keng Ngô le, Ngô we peh ng k'ô nying.''
+	Ge lih-k'eh tön-gao mông, keng Ge k'e.
+	Yia-su jong keh su-ze tseo-ku, yi mông-djôh liang hyüong-di, ziu-z Si-pi-t'a-keh n-ts Ngô-kôh teh ge hyüong-di Iah-'ön, teh ge pang Si-pi-t'a ze jün-li pu mông; ziu ao ge-he.
+	Ge-he lih-k'eh li-k'e jün teh pang, keng Ge.
\par
+	Yia-su tseo-pin Kô-li-li, ze ge-he jü-we-dông kao-hyüing, djün t'in-koh-keh foh-ing, i-hao pah-sing yiang-yiang bing-t'ong.
+	Ge-keh ming-sing yiang-pin Jü-li-ô: væn-pah yiu bing, feh-leng zao-teh mao-bing, t'ong-k'u, bi kyü vu-keh, tin-gao, fong-t'æn, tu ta tao Yia-su su-ze; Yia-su ziu i ge hao.
+	Jong Kô-li-li, Di-kô-po-li, Yia-lu-sæh-lang, Yiu-t'a, teh Iah-dæn 'o-nga, yiu hyü-to nying le keng Ge.



\section{Mt.}%5

\begin{sAbstract}
	\item[1] Kyi-toh ze sæn-zông djün dao:
	\item[3] kông zao væn-ts nying yiu foh,
	\item[13] meng-du z si-zông-keh yin,
	\item[14] si-zông-keh liang-kwông, ih-tu zing ze sæn-zông,
	\item[15] teng-liang,
	\item[17] Kyi-toh le iao tso-zing leh-fæh.
	\item[21] Sæh nying-keh i-s,
	\item[27] kæn-ying-keh i-s,
	\item[33] væh-tsiu-keh i-s;
	\item[38] ky'ün nying kw'ön-shü bih-ke,
	\item[43] e-sih ling-sô,
	\item[48] ke tso nying jün-bi.
\end{sAbstract}

\header
\lettrine{Y}{ia-su} mông-djôh keh hyü-to nying, ziu tseo-zông sæn: ih zo-lôh, meng-du tao Ge sing-pin.\
+	Yia-su k'e-k'eo kao-hyüing ge-he, kông: ---
\par
+	``Sing-li gyüong-keh yiu foh: ing-yü t'in-koh z ge-ke.
\par
+	``Sing-t'ong-keh yiu-foh: ing-yü ge-he we teh-djôh ön-yü.
\par
+	``Weng-ziu-keh yiu-foh: ing-yü ge-he we teh-djôh di-t'u tso sæn-nyih.
\par
+	``Ziang du-hæh k'eo-k'eh s-siang kong-nyi yiu foh: ing-yü ge-he we teh-djôh sing-mön-i-coh.
\par
+	``E-lin-nying-keh yiu foh: ing-yü ge-he we teh-djôh e-lin.
\par
+	``Sing-li ts'ing-kyih yiu foh: ing-yü ge-he we mông-djôh Zông-ti.
\par
+	``Ky'ün nying 'o-moh-keh yiu foh; ing-yü ge-he we ts'ing-hwu z Zông-ti-keh n-ts.
\par
+	``Yü kong-nyi ziu pih-næh-keh, yiu foh: ing-yü t'in-koh z ge-he.
\par
+	``Ziah yiu nying yü Ngô zôh ng-he, pih-næn ng-he, yüong yiang-yiang ôh shih-wa hyü-pông ng-he, ng yiu foh.
\par
+	``Ing-ke hwön-hyi kw'a-lôh: ing-yü ze t'in-zông ng-he-keh pao-ing z ting do: ing-yü ge-he yi z-t'ih pih-næn ng-he zin-deo-keh sin-ts-nying.
\par
+	``\textsc{Ng-he} z si-zông-keh yin: yin ziah-z sih-gao ge 'æn-keh mi-dao, yiu zah-m hao tsông ge 'æn? 'eo-deo z vu-yüong, tsih hao tao ze nga-min peh nying dæh-gao.
\par
+	``Ng-he z si-zông-keh liang-kwông. Ih-tu zing zao ze sæn-zông feh neng ing-zông.
+	Nying tin teng, fe fông ze teo-'ô, fông ze teng-de zông; ziu tsiao-djôh\ 'eh-kô-keh nying.
+	Ng-he-keh liang-kwông ah ing-ke tsiao ze nying min-zin, peh ge mông-djôh ng-keh hao 'ang-yü, ziu we kyü yüong-wa peh ng-he t'in-zông Vu-ts'ing.
\par
+	``Feh-k'o ts'eng Ngô le iao fi-gao leh-fæh teh sin-ts-nying: Ngô le feh-z iao fi-gao, z iao tso-zing.
+	Ngô lao-zih teh ng kông, Teng-tao t'in-di ku-k'e, leh-fæh ih-tin ih-wah feh neng-keo ku-k'e, dzih tao yiang-yiang tso-zing.
+	Ka-ming lin ih-diao ting siao, ziah yiu nying fi-gao, yi kao bih-nying fi-gao, keh nying ze t'in-koh sön ting siao: ziah yiu nying i-jing ge yi kao bih-nying i-jing, keh nying ze t'in-koh sön do.
+	Ngô teh ng kông, Ng-keh kong-nyi ziah feh ku-yü doh-shü-nying teh Fæh-li-se nying-keh kong-nyi, tön feh neng-keo tseo-tsing t'in-koh.
\par
+	``\textsc{Ng-he} t'ing-djôh yiu feng-fu ku-z-tsih nying kông, `Feh-k'o sæh nying; væn-pah hyüong-sæh pih iao sing-p'ön:'
+	dæn-z Ngô teh ng kông, Væn-pah vu-ku ky'i-'eng ge hyüong-di pih iao sing-p'ön; væn-pah zôh-mô ge hyüong-di, Leh-kô, iao song tao kong-dông; væn-pah zôh-mô ge hyüong-di, Mo-li, iao tön ze di-nyüoh-keh ho-li. [\textit{Leh-kô teh Mo-li ziu-z Yiu-t'a nying zôh-mô shih-wa.}]
+	Su-i ng ziah ta li-veh tao tsi-dæn zin, kyi-teh ng yiu teh-ze hyüong-di,
+	ng-keh li-veh fông ze tsi-dæn zin, sin k'e teh hyüong-di 'o-hao, 'eo-deo le hyin li-veh.
\par
+	Ng teh ün-kô wæ ze lu-zông, iao kw'a-ting teh ge kông 'o; sang-teh ün-kô song ng peh kwön, kwön kao peh ts'a-yüoh, ziu kwæn ng ze kæn-lao.
+	Ngô lao-zih teh ng kông, Ziah yiu ih-ke dong-din vong wæn wön, ng tsong feh neng tseo-c'ih.
\par
+	``Ng-he t'ing-djôh yiu feng-fu ku-z-tsih nying kông, `Feh-k'o kæn-ying:'
+	dæn-z Ngô teh ng kông, Væn-pah mông-djôh vu-nyü ky'i ying-sing sing-li z yi-kying væn-djôh kæn-ying.
+	Ng jing-tsih ngæn ziah-z ta-li ng væn-ze, hao leo-c'ih, tön-gao: neng-ts pah-t'i ky'üih ih-yiang, feh iao jün-sing tön-lôh di-nyüoh.
+	Ng jing-tsih siu ziah-z ta-li ng væn-ze, hao tsôh-lôh, tön-gao: neng-ts pah-t'i ky'üih ih-yiang, feh iao jün-sing tön-lôh di-nyüoh.
\par
+	``Yi yiu kông, `Ziah yiu nying iao li-k'e ts'i-ts, ing-ke sia li-shü peh ge:'
+	dæn-z Ngô teh ng kông, Ziah feh-z yü kæn-ying li-k'e ts'i-ts, z 'e ge væn kæn-ying: ziah yiu nying c'ü li-gao-keh ts'i-ts yi z væn kæn-ying.
\par
+	``Ng-he yi t'ing-djôh yiu feng-fu ku-z-tsih nying kông, `Feh-k'o væh kô-tsiu, hyü-nyün peh Cü, tsong iao wæn:'
+	dæn-z Ngô teh ng kông, Tön feh-k'o væh-tsiu; feh-k'o ts-tin t'in væh-tsiu, ing-yü t'in z Zông-ti-keh zo yü:
+	yi feh-k'o ts-tin di væh-tsiu, ing-yü di z Ge-keh dæh-kyiah-teng: yi feh-k'o ts-tin Yia-lu-sæh-lang væh-tsiu, ing-yü Yia-lu-sæh-lang z keh-yü do Wông-ti-keh kying-zing.
+	Yi feh-k'o ts-tin z-keh deo væh-tsiu, ing-yü ng feh neng peh ih-keng deo-fæh pin-bah pin-heh.
+	Dæn-z ng-he-keh shih-wa ing-ke Z, kông z; feh-z, kông feh-z; ziah-z kô-ts'eo z jong ôh-i le.
\par
+	``Ng-he t'ing-djôh kông, `Ngæn ti-wæn ngæn, ngô-ts' ti-wæn ngô-ts':'
+	dæn-z Ngô teh ng kông, Feh-k'o ti-dih ôh-nying: ziah yiu nying tang ng jing-pin cü-kwah, tsi-pin ah hao le-cün peh ge tang.
+	Ziah yiu nying kao ng siang ng-keh li-min i-zông, lin nga-min i-zông ah hao peh ge do-k'e.
+	Ziah yiu nying ngang-æh ng tseo ih-li lu, ziu hao teh ge tseo liang-li.
+	Gyiu ng-keh, hao peh ge; iao tsia ng-keh, feh iao t'e-z.
\par
+	``Ng-he t'ing-djôh kông, `Iao e-sih ng-keh ling-sô, ky'i ng-keh ün-kô:'
+	dæn-z Ngô teh ng kông, Iao e-sih ng-keh ün-kô; coh-foh tsiu ng cü-ts; ky'i ng-keh, iao de ge hao; hyü-pông ng pih-næn ng-keh, iao de ge tao-kao.
+	Ng ziu hao tso ng T'in-Vu-keh n-ts: ing-yü Ge peh Ge-keh nyih-deo tsiao-djôh ôh-nying teh zin-nying; yi kông-lôh yü peh tsing-dzih teh feh tsing-dzih nying.
+	Ng-he ziah-z e-sih e-sih ng-he cü-ts, yiu zah-m pao-ing? Ziu-z siu-din-liang-keh, ky'i feh-z t'ih tso?
+	Ziah-z tæn tsiao-tsih z-keh hyüong-di, yiu zah-m kah-nga hao? Siu-din-liang-keh, ky'i feh-z t'ih tso?
+	Su-i ng-he tso nying ke jün-bi, ziang ng-he T'in-Vu z jün-bi.



\section{Mt.}%6

\begin{sAbstract}
	\item[1] Kyi-toh wæ ze sæn-zông djün dao, ka-shih tso hao-z,
	\item[5] tao-kao,
	\item[14] nyiao-sô bih-ke,
	\item[16] kying-zih.
	\item[19] ze nô su-ze hao k'ông ze-veh,
	\item[24] voh-z Zông-ti teh ze-veh:
	\item[25] ky'ün ge feh iao yü si-zông meh-z iu-zeo:
	\item[33] sin gyiu Zông-ti-keh koh.
\end{sAbstract}

\header
\lettrine{``N}{g-he} tso hao-z iao liu-sing, feh-k'o dih-di ze nying min-zin peh ge mông: ziah-z t'ih ng-he feh neng teh-djôh T'in-Vu-keh pao-ing.
\par
+	``Su-i tso hao-z z-'eo, feh-k'o ze ng min-zin c'ü 'ao-deo, ziang kô-hao-nying ze jü-we-dông teh do-ka-zông su tso, siang teh-djôh nying-keh ts'ing-tsæn. Ngô lao-zih teh ng kông, Ge-keh pao-ing yi-kying teh-djôh.
+	Dæn-z ng tso hao-z, jing siu su tso feh k'o peh tsi siu hyiao-teh:
+	s-teh ng-keh hao-z we ön-cong: ng-keh Vu-ts'ing lin ön-cong su-ze tu mông-djôh Z-kyi we ming-ming pao-ing ng.
\par
+	``Ng tao-kao z-'eo, feh-k'o ziang kô-hao-nying: ge-he hwön-hyi ze jü-we-dông teh s-ts'ô-lu-k'eo gyi-kæn tao-kao, dih-di peh nying mông. Ngô lao-zih teh ng kông, Ge-keh pao-ing yi-kying teh-djôh.
+	Dæn-z ng tao-kao z-'eo, tseo-tsing z-keh iu-zing vông-kæn, meng kwæn-gao, gyiu-gyiu ng-keh Vu-ts'ing lin ön-cong su-ze tu ze-teh: ng Vu-ts'ing ze ön-cong su-ze tu mông-djôh, we ming-ming pao-ing ng.
\par
+	``Ng-he tao-kao feh-k'o kông-ku yi kông, ziang bih-koh-nying: ing-yü ge ts'eng yü shih-wa to pih-ding we t'ing.
+	Ng-he feh-k'o tsiao ge yiang: ing-yü vong gyiu Ge ts-zin ng-he su iao yüong, ng-keh Vu-ts'ing yi-kying hyiao-teh.
+	Ng tao-kao ing-ke t'ih kông: --- \par ``Ngô-he Vu-ts'ing ze t'in-zông, dæn-nyün cong-nying tseng-kying Ng sing-zin-keh ming-deo.
+	Dæn-nyün Ng-keh koh we le. Dæn-nyün Ng-keh ts-i we tso-zing, ze di ziang ze t'in.
+	Ngô-he nyih-nyih-keh liang-zih kying-nying s-peh ngô-he.
+	Sô-min ngô-he tsa, ziang ngô-he sô-min bih-nying-keh tsa.
+	Feh iao peh ngô-he p'ong-djôh mi-'oh, kyiu ngô-he t'eh-c'ih hyüong-ôh. Ing-yü koh-veng neng-kön yüong-wa, si-si-de-de tu z Ng-ke. Ô-meng.
+	``Ing-yü ng-he ziah sô-min bih-nying-keh dzæn-c'ü, ng-he T'in-Vu yi we sô-min ng-he-keh dzæn-c'ü.
+	Ng-he ziah feh k'eng sô-min bih-nying dzæn-c'ü, ng-he T'in-Vu ah feh k'eng sô-min ng-keh dzæn-c'ü.
\par
+	``Ng-he kying-zih z-'eo, feh-k'o ziang kô-hao-nying, ta iu-zeo væn-ts: ing-yü ge-he pin-wön min-seh, dih-di peh nying mông-djôh ge ze-teh kying-zih. Ngô lao-zih teh ng kông, Ge-he pao-ing yi-kying teh-djôh.
+	Dæn-z ng kying-zih, yüong s deo, si min;
+	feh iao peh nying mông, tsih iao peh ng ön-cong-keh Vu-ts'ing, lin ön-cong su-ze tu mông-djôh, we ming-ming pao-ing ng.
\par
+	``\textsc{Ng-he} feh-k'o tsih-djü ze-veh ze di-zông, cü-djong iao cü-gao, iao fæh-siu me-læn, dao-zeh we ts'ah dong t'eo-k'e:
+	iao tsih-djü ze-veh ze t'in-zông, cü-djong cü feh-gao, fe fæh-siu me-læn, dao-zeh fe ts'ah dong t'eo-k'e.
+	Ing-yü ng-keh ze-veh ze 'ah-yi, ng-keh sing ah ze 'ah-yi.
+	``Ngæn z sing-zông-keh teng: ng-keh ngæn ziah-z liang, mön-sing yiu kwông.
+	Ng-keh ngæn ziah feh liang, mön-sing z heh-ön. Ng-keh kwông ziah-z heh-ön, keh heh-ön z ting do!
\par
+	``Ih-ke nying feh neng voh-z liang-ke cü-nying-kô: 'oh-tsia we k'o-u ih-ke, e-kying bih-ke; 'oh-tsia we ts'ing-gying i-pin, k'ön-ky'ing kæn-pin. Zông-ti teh ze-veh ng feh neng tu voh-z.
+	Su-i Ngô teh ng kông, Feh-k'o yü sing-ming iu-zeo, zah-m ky'üoh, zah-m hæh; yi feh-k'o yü sing-t'i iu-zeo, zah-m tsiah. Sing-ming ky'i feh-z pi ky'üoh-zih kyü-djong, sing-t'i ky'i feh-z pi i-zông iao-kying?
+	Hao mông k'ong-cong-keh tiao, ge feh cong-lôh, feh siu-keh, yi feh siu-tsing koh-ts'ông-li; ng-keh T'in-Vu we yiang ge. Ng-he ky'i feh-z pi ge wæ kyü-djong?
+	Ng-he ziah-z iu-zeo, nô ih-ke neng-keo to tso ih-k'eh nying?
+	Ng tsa-sang zeo i-zông tsiah? Hao s-ts'eng hông-di-keh pah-'eh hwa, tsa-sang do-ky'i-le; ge feh lao-loh, yi feh fông-min:
+	Ngô teh ng kông, Ziu-z Su-lo-meng ting yüong-wa z-'eo ge tsiah-sih feh gyih-jü ih-tô pah-'eh-hwa.
+	Yia-ts'ao kying-nying wæ ze-teh, t'in-nyiang tön ze ho-lu, Zông-ti we z-t'ih tang-pæn ge, 'o-hyüông ng-he, siao siang-sing-keh nying?
+	Su-i feh yüong iu-zeo, kông, Ngô-he yiu zah-m ky'üoh? zah-m hæh, zah-m tsiah?
+	(Keh-sih tu z bih-koh-nying su gyiu:) ng ih-ts'ih su iao, ng-keh T'in-Vu yi-kying hyiao-teh.
+	Ng ing-ke sin gyiu Zông-ti keh koh, teh Ge kong-nyi; keh ih-ts'ih-keh tu we kô-ts'eo peh ng.
+	Su-i feh-k'o yü t'in-nyiang iu-zeo: t'in-nyiang we zeo t'in-nyiang z-kön. Ih-nyih kwön ih-nyih lao-loh keo-gao.



\section{Mt.}%7

\begin{sAbstract}
	\item[1] Kyi-toh wæ ze sæn-zông djün-dao, tsah-væh p'i-bing bih-ke cü-ts.
	\item[6] sing-zin-keh meh-z m-c'ü tön peh ts,
	\item[7] ky'ün nying tao-kao,
	\item[13] tseo-tsing 'æh-tsæh-keh meng,
	\item[15] vông-bi kô sin-ts-nying.
	\item[21] feh tæn-tsih iao t'ing, yi iao i-jing:
	\item[24] oh ky'i ze zih-deo zông,
	\item[28] oh ky'i ze sô-t'æn-zông. 
\end{sAbstract}

\header
\lettrine{``N}{g-he} feh-k'o p'i-bing nying, sang-teh nying p'i-bing ng.
+	Ing-yü ng-he p'i-bing nying tsa-sang, p'i-bing ng ah we tsa-sang: ng-he liang peh bih-nying tsa-sang, liang peh ng ah we tsa-sang.
+	Ng yü zah-m mông-djôh hyüong-di ngæn-li yiu ts', feh ts'eng-tao z-keh ngæn-li yiu tong-liang?
+	Ng tsa-sang hao teh hyüong-di kông, `Ng ngæn-li-keh ts' peh ngô bæh-c'ih;' ng-z ngæn-li yiu tong-liang?
+	Kô-hao-nying, sin bæh-c'ih ng-z ngæn-li-keh tong-liang; 'eo-deo ng neng-keo mông-ming, hao bæh-c'ih hyüong-di ngæn-li-keh ts'.
\par
+	``Sing-zin-keh meh-z feh-k'o peh keo, ng keh tsing-cü yi feh-k'o tön peh ts, sang-teh ge kyiah dæh-gao, wæ iao le-cün ngao ng.
\par
+	``Ng-he gyiu, tsong we peh ng; zing, tsong we zing-djôh; k'ao meng, tsong we k'e peh ng:
+	ing--yü væn-pah gyiu-keh we teh-djôh; zing-keh, we zing-djôh; k'ao-meng-keh we k'e peh ge.
+	Ng-he cong-yiang nô ih-ke yiu n-ts gyiu ih-ke ping, we côh ih-kw'e zih-deo peh ge;
+	gyiu ih-kwang ng, we côh ih-kwang zô peh ge?
+	Ng-he shü-tsih z ôh-nying, ziah hyiao-teh yüong hao meh-z peh n-nô, 'o-hyüông ng-he t'in-zông Vu-ts'ing ky'i feh kah-nga yüong hao meh-z s-peh gyiu Ge cü-ts?
\par
+	``Su-i ng-he iao bih-nying tsa-sang de ng, ng ah iao tsa-sang de ge: keh ziu-z leh-fæh  teh sin-ts-nying-keh i-s.
\par
+	``\textsc{Ng-he} iao tseo-tsing 'æh-tsæh-keh meng: ing-yü tao djü-mih su-ze, meng z kw'eh, lu z do, tseo-tsing-keh nying to.
+	Dæn-z tao weh-ming di-fông, meng z 'æh-tsæh, lu z siao, zing-djôh cü-ts feh to.
\par
+	``Ng-he kying-vông kô sin-ts-nying, ge tao ng su-ze le, nga-min ziang yiang, li-min z za-lông.
+	Ng-he mông ge ko-ts hao hyiao-teh ge. Ts'-bang cong-yiang 'ah-yi yiu bu-dao tsah, zih-li cong-yiang 'ah-yi yiu vu-hwa-ko tsah?
+	Su-i, hao jü we kyih hao ko-ts; feh hao jü we kyih feh hao ko-ts.
+	Hao jü feh neng kyih feh hao ko-ts, feh hao jü feh neng kyih hao ko-ts.
+	Væn-pah feh kyih hao ko-ts-keh jü, tu tsôh-gao tön ze ho-li.
+	Su-i mông ge ko-ts hao hyiao-teh ge.
\par
+	``Bing fi væn-pah ts'ing-hwu Ngô, `Cü, Cü,' hao tseo-tsing t'in-koh; tsih-yiu i Ngô t'in-zông Vu-ts'ing-keh ts-i hao tseo-tsing.
+	Tao keh nyih yiu hyü-to nying we teh Ngô kông, `Cü, Cü, ngô-he ky'i feh-z vong Ng-keh ming-deo djün-dao, vong Ng-keh ming-deo kön-c'ih kyü, vong Ng-keh ming-deo tso hyü-to gyi-z?'
+	Ngô ziu we ming-ming teh ge kông, Ngô jong-le feh nying-teh ng: ng-he tso fi-li z-kön hao li-k'e Ngô.
\par
+	Su-i væn-pah t'ing Ngô keh-sih shih-wa, ziu i ge tso, k'o-pi ts'ong-ming-keh nying ky'i oh ze zih-deo-zông:
+	yü lôh-le, shü mön-zông, fong c'ü-ku, tu djông-djôh keh oh; ge tao-feh-gao: ing-yü ziang-kyiah fông ze ngæn-zih zông.
+	Væn-pah t'ing-djôh Ngô keh-sih shih-wa, feh i ge tso, k'o-pi ih-ke nge-beng nying, ky'i oh ze sô-t'æn-zông:
+	yü lôh-le, shü mön-zông, fong c'ü-ku, tu djông-djôh keh oh; ge ziu tao-gao: ge-keh tao-gao z li-'e.''
\par
+	Yia-su shih-wa kông wön, cong-nying tu hyi-gyi Ge-keh kao-hyüing:
+	ing-yü kao-hyüing ge-he ziang yiu gyün-ping, feh ziang doh-shü-nying.



\section{Mt.}%8

\begin{sAbstract}
	\item[2] Kyi-toh i da-mô-fong,
	\item[5] i pah-tsong-keh yüong-nying,
	\item[14] i Pi-teh-keh lao-dziang-m,
	\item[16] i hyü-to bing-nying;
	\item[18] keng Cü-keh i-s:
	\item[23] peh do fong bing-zing,
	\item[28] jong liang-ke nying sing-zông kön-c'ih kyü,
	\item[31] ing-zing kyü tseo-tsing ih-pæn ts cong-yiang.
\end{sAbstract}

\header
\lettrine{Y}{ia-su} tseo-lôh sæn, yiu hyü-to nying keng Ge.
\par
\hspace{1ex}
+	\textsc{Ziu} yiu ih-ke nying sang da-mô-fong le pa Ge, kông, ``Cü, Ng ziah-z k'eng, neng-keo peh ngô kyih-zing.''
+	Yia-su long-c'ih siu, môh-môh ge, kông, ``Ngô z k'eng ng hao kyih-zing.'' Da-mô-fong lih-k'eh kyih-zing.
+	Yia-su teh ge kông, ``Feh-k'o teh bih-nying kông; tæn-tsih k'e peh tsi-s mông-ih-mông, tsiao Mo-si-keh feng-fu hyin li-veh, hao peh ge-he tso te-tsing.''
\par
+	Yia-su tseo-tsing Kô-pah-nong z-'eo, yiu ih-ke pah-tsong tseo-le gyiu Ge, kông,
+	``Cü, ngô-keh yüong nying sang fong-t'æn bing tao ze u-li, t'ong-teh mang.''
+	Yia-su teh ge kông, ``Ngô we le i ge.''
+	Pah tsong we-teh kông, ``Cü, Ng tao ngô sô-ô le ngô feh kön-tông: Ng tsih kông ih-kyü, ngô-keh yüong-nying ziu we jün-yü.
+	Ing-yü ngô z voh bih-nying ke-kwön, yi yiu ping voh ngô ke-kwön: ngô ao ih-ke k'e, ziu k'e, ao ih-ke le, ziu le; ao ngô nu-boh, k'e tso ih-yiang z-kön, ge ziu k'e tso.''
+	Yia-su t'ing-djôh ziu hyi-gyi, teh keng-jong nying kông, ``Ngô lao-zih teh ng kông, t'ih do siang-sing-keh sing, ziu-z ze Yi-seh-lih pah-sing cong-yiang Ngô vong p'ong-djôh-ku.
+	Ngô yi teh ng kông, Jong tong-pin jong si-pin yiu hyü-to nying we le, teh Ô-pah-læh-hön, Yi-sæh, Ngô-kôh, ze t'in-koh jü-de zo-zih:
+	dæn-z peng-koh-keh n-ts iao kön-c'ih tao nga-min heh-ön di-fông: ze-kæn pih yiu ngao-ngô-ts'ih-ts' k'oh-hyiang.''
+	Yia-su teh pah-tsong kông, ``Hao cün-k'e; tsiao ng siang-sing-keh sing peh ng tso-zing.'' Ge-keh yüong-nying k'eo-k'eo keh z-'eo jün-yü.
\par
+	Yia-su tseo-tsing Pi-teh u-li, mông ge lao-dziang-m kyi-sing fæh nyih tao-kæn.
+	Ziu môh-môh ge siu, nyih bing ziu t'e-gao; ge ziu nang-ky'i-le voh-z Ge.
\par
+	Æn-keng yiu nying ta hyü-to bi kyü vu-keh nying tao Yia-su su-ze: Yia-su yüong ih-kyü wa kön-c'ih kyü, wæ-yiu i-hao ih-ts'ih sang-bing-keh nying:
+	hao peh sin-ts Yi-se-ô su kông shih-wa yiu ing-nyin, ziu-z, ``Ge z tæn-k'e ngô-he-keh k'u-næn, tæn-tông ngô-he-keh bing-t'ong. [Y. 53. 4]
\par
+	\textsc{Yia-su} mông-djôh hyü-to nying yü-cün Ge, ziu feng-fu meng-du tao te-'ön k'e.
+	Yiu ih-ke doh-shü-nying tseo-le, teh Ge kông, ``Sin-sang, Ng feh-leng tao 'ah-yi k'e ngô iao keng Ng.''
+	Yia-su teh ge kông, ``Wu-li yiu dong, k'ong-cong-keh tiao yiu k'o; dæn-z Nying-keh N-ts m-yiu su-ze hao hjü.''
+	Yi yiu ih-ke meng-du teh Ge kông, ``Cü, peh ngô sin cün-k'e ön-tsông ngô ah-pang.''
+	Yia-su teh ge kông, ``Keng Ngô le; peh s-nying ön-tsông ge s-nying.
\par
+	Yia-su lôh jün, meng-du ziu keng Ge.
+	He-zông mah-ding tsôh do fong-lông, jün bi lông ke-lao: dæn-z Yia-su kw'eng-kæn.
+	Meng-du tseo-le, ao Ge sing, kông, ``Cü, kyiu ngô-he: ngô-he iao s.''
+	Yia-su teh ge kông, ``Siao siang-sing-keh nying, ng yü zah-m p'ô?'' Ziu nang-ky'i, tsah-væh fong teh he; he lih-k'eh bing-zing.
+	Cong-nying tu hyi-gyi, kông, ``Keh-z zao-teh Nying, lin fong teh he tu we i-jing Ge?''
\par
+	Yia-su ih tao te-'ön Kah-kah-sô di-fông, yiu liang-ke bi kyü vu-keh nying ting li-'e, nying m-tæn tseo-ku keh-da lu, jong veng-mo tseo-c'ih p'ong-djôh Ge.
+	Ziu ao-ky'i-le, kông, ``Zông-ti-keh N-ts Yia-su, ngô-he teh Ng yiu zah-m siang-kön? z-'eo vong tao Ng we tao-i le mo-næn ngô-he?''
+	Yün su-ze yiu ih-do-pæn ts ze-kæn ky'üoh.
+	Kyü ziu gyiu Yia-su, kông, ``Ng ziah kön ngô-he c'ih, peh ngô-he tseo-tsing ts cong-yiang.''
+	Yia-su teh ge kông, ``K'e.'' Ge-he tseo-c'ih, tao ts cong-yiang: 'eh-pæn ts diao tao sæn-kyiah ts'ön-lôh he-li, tsing-s.
+	K'ön ts nying diao-tsing zing, keh-gyin z-kön, lin kyü vu-keh z-kön, tu t'ong-ts ge-he.
+	Zing-li nying tu tseo-c'ih, ih mông-djôh Yia-su, ziu gyiu Ge li-k'e ge di-ka.



\section{Mt.}%9

\begin{sAbstract}
	\item[2] Kyi-toh i fong-t'æn,
	\item[9] ao zo ze shü-kwæn-keh Mô-t'a keng Ge,
	\item[10] teh siu-din-liang ze-nying jü-de zo-zih,
	\item[14] kông zao yün-ku Ge-keh meng-du feh kying-zih,
	\item[20] i hyüih-leo-keh bing,
	\item[23] peh Nga-lu-keh nô weh-cün-le,
	\item[27] i liang-ke hæh-ngæn,
	\item[32] i hao bi kyü vu-keh ô-lao,
	\item[36] e-lin si-zông nying.
\end{sAbstract}

\header
\lettrine{Y}{ia-su} lôh jün, ku-du tao Z-keh peng-zing.
\par
\hspace{1ex}
+	Yiu nying kông ih-ke fong-t'æn, kw'eng ze p'u-pæn, tao Ge su-ze: Yia-su mông ge-he yiu siang-sing-keh sing, teh fong-t'æn cü-ts kông, ``N-ts, fông-sing; ng-keh ze sô-gao.''
+	Yiu kyi-ke doh-shü-nying sing-li kông, ``Keh Nying kông sih-doh shih-wa.''
\par
+	Yia-su hyiao-teh ge-keh sing-i ziu kông, ``Ng-he sing-li tsa-sang zeng ôh-i?
+	Wæ-z kông, ``Ng-keh ze sô-gao, wæ-z kông, Nang-ky'i tseo,--- nô ih-yiang yüong-yi?
+	Yin-ze peh ng-he hyiao-teh Nying-keh N-ts ze si-zông yiu sô-ze-keh gyün-ping, (ziu teh fong-t'æn cü-ts kông,) Nang-ky'i, do p'u-pæn cün u-li k'e.''
+	Ge ziu nang-ky'i, cün u-li k'e.
+	Cong-nying mông-djôh tu hyi-gyi, ziu kyü yüong-wa peh Zông-ti, ing-yü Ge yiu t'ih do gyün-ping s-peh nying.
\par
+	\textsc{Yia-su} jong keh su-ze tseo-ku, mông-djôh ih-ke nying, ming-z Mô-t'a, zo ze shü-kwæn: teh ge kông, ``Keng Ngô le.'' Ge ziu nang-ky'i, keng Yia-su.
\par
+	Yia-su ze ge oh-li zo-zih, yiu hyü-to siu-din-liang teh ze-nying tseo-le, teh Yia-su lin meng-du jü-de zo-zih.
+	Fæh-li-se nying mông-djôh, teh Ge meng-du kông, ``Ng-keh Sin-sang tsa-sang teh siu-din-liang lin ze-nying jü-de ky'üoh-væn?''
+	Yia-su t'ing-djôh, teh ge kông, ``Gyin-keh nying feh iao t'a-i sin-sang, yiu bing nying iao.
+	`Ngô iao e-lin, feh iao tsi-veh,'['O. 6. 6.] keh-tsih shü tsa-sang i-s, ng-he hao k'e ts'eng ming-bah: ing-yü Ngô le, feh-z ao tsing-dzih nying hwe-ke, z ao ze-nying hwe-ke.''
\par
+	Keh z-'eo Iah-'ön-keh meng-du tao Yia-su su-ze le, kông, ``Ngô-he teh Fæh-li-se nying zông-zông kying-zih, Ng-keh meng-du tsa-sang feh kying-zih?''
+	Yia-su teh ge kông, ``Sing-lông teh 'o-hyi nying jü-de, tsa-sang iao sing-t'ong? dæn-z tsiang-le sing-lông iao li-k'e ge-he, keh z-'eo iao kying-zih.
+	M-yiu nying yüong sing pu pu gyiu i-zông; sang-teh su pu-keh sing pu pang-se gyiu i-zông, ge-keh se-k'ong ziu kah-nga do.
+	Ah feh yüong gyiu bi-de din sing tsiu: sang-teh bi-de lih-k'e, tsiu leo-gao, bi-de yi wa-gao: tsong z yüong sing bi-de din sing tsiu, liang-yiang tu pao-lao.''
\par
+	\textsc{Yia-su} teh ge-he kông keh shih-wa z-'eo, yiu ih-ke kwön le pa Ge, kông, ``Ngô-keh nô te s-gao: dæn-z Ng le yüong siu en ge, ge we weh.''
+	Yia-su ziu ky'i-le, keng ge, meng-du ah jü-de k'e.
\par
+	Yiu ih-ke nyü-nying sang hyüih-leo bing zih-nyi nyin, ze Yia-su 'eo-min tseo-le, môh-djôh Ge i-zông-kying:
+	ing-yü ge sing-li kông, ``Ngô tæn-tsih môh-djôh Ge i-zông, ziu jün-yü.''
+	Yia-su le-cün mông-djôh ge, kông, ``Nô, ng hao fông-sing; z ng siang-sing-keh sing peh ng jün-yü.'' Keh nyü-nying tông-z jün-yü.
\par
+	Yia-su ih tao kwön-keh u-li, mông-djôh yiu c'ü-ku-tang, wæ-yiu hyü-to nying yia-yia-sing,
+	ziu teh ge kông, ``Tseo-k'e ting: keh nô vong s, z kw'eng-k'e.'' Ge-he ziu lang-siao Ge.
+	Cong-nying bi kön-c'ih, Yia-su ziu tseo-tsing, ky'in ge siu; keh nô ziu nang-ky'i.
+	Keh z-kön yiang-pin keh di-fông.
\par
+	Yia-su jong keh di-fông tseo-ku, yiu liang-ke hæh-ngæn keng Ge, ao-hyiang, kông, ``Da-bih-keh 'Eo-de, k'o-lin ngô-he.''
+	Yia-su tseo-tsing oh-li, hæh-ngæn tseo tao Ge min-zin: Yia-su teh ge kông, ``Keh z-kön ng-he siang-sing Ngô neng-keo tso feh?'' Ge kông, ``Cü, ngô-he siang-sing.''
+	Yia-su ziu môh-môh ge ngæn, kông, ``Tsiao ng siang-sing-keh sing peh ng tso-zing.''
+	Ge-keh ngæn ziu mông-djôh. Yia-su coh-fu ge, kông, ``Yüong liu-sing feh-k'o peh bih-nying hyiao-teh.''
+	Dæn-z ge-he tseo-c'ih peh Yia-su ming-sing djün-pin keh di-fông.
\par
+	Ge-he tseo-c'ih z-'eo, yiu nying ta ih-ke bi kyü vu-keh ô-lao tao Ge min-zin.
+	Kyü ih kön-c'ih, ô-lao ziu kông shih-wa: cong-nying tu hyi-gyi, kông, ``Ze Yi-seh-lih di-fông jong-le vong mông-djôh keh-yiang z-kön.''
+	Dæn-z Fæh-li-se nying kông, ``Ge z k'ao-djôh kyü-wông kön-c'ih kyü.''
\par
+	\textsc{Yia-su} tseo-pin kôh zing-li kôh hyiang-ts'eng, ze ge-he jü-we-dông kao hyüing, djün t'in-koh-keh foh-ing, wæ-yiu i pah-sing-keh yiang-yiang bing-t'ong.
+	Mông-djôh keh hyü-to nying, ziu fæh z-pe sing e-lin ge-he, ing-yü ge-he z kw'eng-k'u li-sæn, ziang m-nying kwön-keh yiang.
+	Ge ziu teh meng-du kông, ``Nyin-zing da-joh, tso kong-nying ky'üih.
+	Su-i ng-he ke gyiu siu-keh Cü-nying-kô ts'a kong-nying siu-keh.''



\section{Mt.}%10

\begin{sAbstract}
	\item[1] Kyi-toh ts'a zih-nyi s-du c'ih-meng djün-dao i bing,
	\item[5] feng-fu ge-he tao 'ah-yi k'e, tsa tso-fæh,
	\item[16] t'ong-ts ge iao p'ong-djôh pih-næn z-kön:
	\item[40] Kông tsih-ziu ge-he cü-ts we teh-djôh pao-ing.
\end{sAbstract}

\header
\lettrine{Z}{iu} ao Ge zih-nyi meng-du le, s-peh ge gyün-ping hao kön-c'ih we-u-keh kyü, i hao yiang-yiang bing-t'ong.
\par
+	Keh zih-nyi s-du ming-z di-ih, Si-meng yi ts'ing-hwu Pi-teh teh ge hyüong-di Ön-teh-lih; Si-pi-t'a-keh n-ts Ngô-kôh, teh ge hyüong-di Iah-'ön;
+	Fi-lih, teh Pô-to-lo-ma; To-mô, teh siu-din-liang-keh Mô-t'a; Ô-læh-fi-keh n-ts Ngô-kôh, teh Læh-pa, yi ts'ing-hwu Dæh-t'a;
+	Kô-nô-keh Si-meng, teh ma Yia-su Kô-liah-keh Yiu-da.